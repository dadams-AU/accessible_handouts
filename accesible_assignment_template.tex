\DocumentMetadata{
  pdfstandard=UA-2,
  pdfversion=2.0,
  lang=en-US,
  title={ASSIGNMENT TITLE},
  author={INSTRUCTOR NAME},
  subject={COURSE CODE -- COURSE TITLE},
  keywords={CSUF, Accessible Handout, Course Materials}
}

\documentclass[12pt]{article}

\usepackage[american]{babel}

% ================================
% FONT SETTINGS
% ================================
\usepackage{fontspec}
\IfFontExistsTF{TeX Gyre Heros}{
  \setmainfont{TeX Gyre Heros}
}{
  \setmainfont{Latin Modern Sans}
}
\IfFontExistsTF{TeX Gyre Cursor}{
  \setmonofont{TeX Gyre Cursor}
}{
  \setmonofont{Latin Modern Mono}
}

% Page layout and spacing
\usepackage{geometry}
\geometry{margin=1in}
\usepackage{setspace}
\onehalfspacing

% Lists, tables, graphics
\usepackage{enumitem}
\setlist[itemize]{itemsep=0pt, parsep=0pt, topsep=0.25\baselineskip}
\newlist{flatlist}{itemize}{1}
\setlist[flatlist]{label={}, leftmargin=0pt, itemsep=0pt, parsep=0pt, topsep=0pt}
\setlistdepth{3}
\usepackage{graphicx}
\usepackage{array, booktabs, longtable}
\usepackage{caption}
\usepackage{tabularray}
\UseTblrLibrary{booktabs}

% Tagging for accessibility
\IfFileExists{tagpdf-base.sty}{
  \usepackage{tagpdf}
  \tagpdfsetup{activate-all=true, interwordspace=true}
}{
  \newcommand\tagpdfsetup[1]{}
}

% Header/footer
\usepackage{fancyhdr}
\pagestyle{fancy}
\fancyhf{}
\fancyhead[R]{COURSE CODE -- COURSE TITLE}
\fancyhead[L]{\textit{ASSIGNMENT TITLE}}
\fancyfoot[C]{\thepage}

% Links and metadata
\usepackage{hyperref}
\usepackage{bookmark}
\hypersetup{
  colorlinks=true,
  linkcolor=blue,
  citecolor=magenta,
  urlcolor=blue,
  unicode=true,
  pdftitle={ASSIGNMENT TITLE},
  pdfauthor={INSTRUCTOR NAME},
  pdfsubject={COURSE CODE -- COURSE TITLE},
  pdfkeywords={CSUF, Accessible Handout, Course Materials},
  pdfborderstyle={/S/U/W 1}
}
\urlstyle{same}

% Bibliography (optional)
\usepackage[round]{natbib}

% Pandoc compatibility
\providecommand{\tightlist}{}

% Document title block
\title{ASSIGNMENT TITLE\\\large COURSE CODE -- COURSE TITLE}
\author{INSTRUCTOR NAME}
\date{TERM AND YEAR}

\begin{document}
\maketitle

% ========== OPTIONAL: LOGO ==========
% \begin{center}
% \includegraphics[width=2.25in, alt={Cal State Fullerton wordmark}]{csuf_logo.png}
% \end{center}

\section*{Purpose}
Briefly explain what students will learn or practice.

\section*{Overview}
Summarize the assignment and how it fits the course.

\section*{Learning Objectives}
\begin{itemize}
\item Objective 1.
\item Objective 2.
\item Objective 3.
\end{itemize}

\section*{Requirements}
\begin{itemize}
\item Requirement 1.
\item Requirement 2.
\item Requirement 3.
\end{itemize}

\section*{Deliverables}
\begin{itemize}
\item Deliverable 1 (format, length, file type).
\item Deliverable 2.
\end{itemize}

\section*{Timeline}
\begin{itemize}
\item \textbf{Due date:} DAY, DATE, TIME.
\item \textbf{Milestones:} List any checkpoints.
\end{itemize}

\section*{Evaluation Criteria}
\begin{itemize}
\item Criterion 1 (weight or points).
\item Criterion 2 (weight or points).
\item Criterion 3 (weight or points).
\end{itemize}

\section*{Submission Instructions}
Where and how to submit (Canvas link, file name, etc.).

\section*{Accessibility and Support}
Invite students to contact you for accommodations and list support resources.

\end{document}
